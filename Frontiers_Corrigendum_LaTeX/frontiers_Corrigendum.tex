%%%%%%%%%%%%%%%%%%%%%%%%%%%%%%%%%%%%%%%%%%%%%%%%%%%%%%%%%%%%%%%%%%%%%%%%%%%%%%%%%%%%%%%%%%%%%%%%%%%%%%%%%%%%%%%%%%%%%%%%%%%%%%%%%%%%%%%%%%%%%%%%%%
% This is just an example/guide for you to refer to when submitting manuscripts to Frontiers, it is not mandatory to use Frontiers .cls files nor frontiers.tex  %
% This will only generate the Manuscript, the final article will be typeset by Frontiers after acceptance.                                                 %
%                                                                                                                                                         %
% When submitting your files, remember to upload this *tex file, the pdf generated with it, the *bib file (if bibliography is not within the *tex) and all the figures.
%%%%%%%%%%%%%%%%%%%%%%%%%%%%%%%%%%%%%%%%%%%%%%%%%%%%%%%%%%%%%%%%%%%%%%%%%%%%%%%%%%%%%%%%%%%%%%%%%%%%%%%%%%%%%%%%%%%%%%%%%%%%%%%%%%%%%%%%%%%%%%%%%%

%%% Version 3.2 Generated 2016/11/10 %%%
%%% You will need to have the following packages installed: datetime, fmtcount, etoolbox, fcprefix, which are normally inlcuded in WinEdt. %%%
%%% In http://www.ctan.org/ you can find the packages and how to install them, if necessary. %%%

\documentclass[utf8]{frontiers_correction} 


%\setcitestyle{square} % for Physics and Applied Mathematics and Statistics articles
\usepackage{url,hyperref,lineno,microtype}
\usepackage[onehalfspacing]{setspace}
\linenumbers


% Leave a blank line between paragraphs instead of using \\


\def\keyFont{\fontsize{8}{11}\helveticabold }
\def\firstAuthorLast{Sample {et~al.}} %use et al only if is more than 1 author
\def\Authors{First Author\,$^{1,*}$, Co-Author\,$^{2}$ and Co-Author\,$^2$}

%%%%%%%%%%%%%%%%%%%%%%%%%%%%%%%%%%%%%%%%%%%%%%%%%%%%%%%%%%%%%%
%%%%%%%%%%%%%%%%%%%%%%%%%%%%%%%%%%%%%%%%%%%%%%%%%%%%%%%%%%%%%%%%%%%%%%%%%%%
%%% Name of all authors should be as they appear in the published original article (If correcting author names please put the correct spelling. If adding authors, please include additional authors.)

%%% Affiliations should be keyed to the author's name with superscript numbers and be listed as follows: Laboratory, Institute, Department, Organization, City, State abbreviation (USA, Canada, Australia), and Country (without detailed address information such as city zip codes or street names).
%%% Affiliations of all authors must be the same as they appear in the published originial version of the article. (If correcting affiliations please put the correct version. If adding affiliations, please include additional affiliations).

%%% If one of the authors has a change of address, list the new address below the correspondence details using a superscript symbol and use the same symbol to indicate the author in the author list.
%%%%%%%%%%%%%%%%%%%%%%%%%%%%%%%%%%%%%%%%%%%%%%%%%%%%%%%%%%%%%%%%%%%%%%%%
%%%%%%%%%%%%%%%%%%%%%%%%%%%%%%%%%%%%%%%%%%%%%%%%%%%%%%%%%%%%%%%%%%%%
\def\Address{$^{1}$Laboratory X, Institute X, Department X, Organization X, City X , State XX (only USA, Canada and Australia), Country X \\
$^{2}$Laboratory X, Institute X, Department X, Organization X, City X , State XX (only USA, Canada and Australia), Country X  }
% The Corresponding Author should be marked with an asterisk
% Provide the email of the corresponding author as it appears in the published original version of the article.
\def\corrAuthor{Corresponding Author}
\def\corrEmail{email@uni.edu}




\begin{document}
\onecolumn
\firstpage{1}

\title[Corrigendum]{Corrigendum: [TITLE OF ORIGNINAL ARTICLE]} % this must be the full title of the origninal article

\author[\firstAuthorLast ]{\Authors} %This field will be automatically populated
\address{} %This field will be automatically populated
\correspondance{} %This field will be automatically populated

\extraAuth{}% If there are more than 1 corresponding authors, comment this line and uncomment the next one.
%\extraAuth{corresponding Author2 \\ Laboratory X2, Institute X2, Department X2, Organization X2, Street X2, City X2 , State XX2 (only USA, Canada and Australia), Zip Code2, X2 Country X2, email2@uni2.edu}


\maketitle

{\tiny
 {\keyFont{ \*{Keywords:} Text Text Text Text Text Text Text Text }}} % These must be taken from the original version of the article. 


\subsection*{A corrigendum on \textit{full citation of the origninal version of the article.}} % Insert full citation from origninal version of the article.

%%%%%%%%%%%%%%%%%%%%%%%%%%%%%%%%%%%%%%%%%%%%%%%%%%%%%%%%%%%%%%%%%%%%%%%%%%%%
%%% PLEASE PICK THE MOST RELEVANT TEXT TEMPLATE(S) (delete headings and unneccesary templates) AND EDIT AS NECESSARY.
%%%%%%%%%%%%%%%%%%%%%%%%%%%%%%%%%%%%%%%%%%%%%%%%%%%%%%%%%%%%%%%%%%%%%%%%%%%%%%

\subsubsection*{Incorrect Funding Number}
There is an error in the Funding statement. The correct number for [Insert Funder] is [Insert Grant Number]. The authors apologize for this error and state that this does not change the scientific conclusions of the article in any way.

\subsubsection*{Incorrect Funder Name}
There is an error in the Funding statement. The correct Name for the Funder is [Insert CORRECT Name]. The authors apologize for this error and state that this does not change the scientific conclusions of the article in any way.

\subsubsection*{Missing Funding}
In the original article, we neglected to thank our funder [Insert Funder], [Insert Grant Number] to [Name of Author]. The authors apologize for this error and state that this does not change the scientific conclusions of the article in any way.

\subsubsection*{Figure/Table Legend}
In the original article, there was a mistake in the legend for [Name of Figure/Table] as published. [State the mistake which was made]. The correct legend appears below. The authors apologize for this error and state that this does not change the scientific conclusions of the article in any way.
[Insert CORRECT legend]

\subsubsection*{Error in Figure/Table}
In the original article, there was a mistake in [Name of Figure/Table] as published. [State the mistake which was made]. The corrected [Name of Figure/Table] appears below. The authors apologize for this error and state that this does not change the scientific conclusions of the article in any way.

\subsubsection*{Incorrect Author Name}
An author name was incorrectly spelled as [Insert incorrect version]. The correct spelling is [Insert CORRECT version]. The authors apologize for this error and state that this does not change the scientific conclusions of the article in any way.

\subsubsection*{Incorrect Affiliation}
In the published article, there was an error in Affiliation [Insert Number]. Instead of “[Insert incorrect version]”, it should be “[Insert correct version]”. The authors apologize for this error and state that this does not change the scientific conclusions of the article in any way.

\subsubsection*{Addition of Affiliation}
In the published article, there was an error regarding the affiliation[s] for [Name of Author]. As well as having Affiliation [Insert Number(s)], they should also have [Insert affiliations]. The authors apologize for this error and state that this does not change the scientific conclusions of the article in any way.

\subsubsection*{Addition of an Author}
[Insert Author Name] was not included as an author in the published article. The authors apologize for this error and state that this does not change the scientific conclusions of the article in any way.

\subsubsection*{Incorrect Reference}
In the original article, the reference for [Insert citation] was incorrectly written as [Insert full reference]. It should be [Insert CORRECT reference]. The authors apologize for this error and state that this does not change the scientific conclusions of the article in any way.

\subsubsection*{Author Name Misspelled in Reference}
In the original article, the name of one of the authors was incorrectly spelled in the reference for [Insert citation] as [Insert incorrect spelling]. It should be [Insert CORRECT spelling]. The authors apologize for this error and state that this does not change the scientific conclusions of the article in any way.

\subsubsection*{Missing Citation}
In the original article [Insert citation] was not cited in the article. The citation has now been inserted in [Name of Section], [Name of Sub-section if there is one], [Paragraph Number] and should read:
[Insert CORRECTED PARAGRAPH]
The authors apologize for this error and state that this does not change the scientific conclusions of the article in any way.

\subsubsection*{Text Correction}
In the original article, there was an error. [State the mistake that was made].
A correction has been made to [Name of Section], [Name of Sub-section if there is one], [Paragraph Number]:
[Insert CORRECTED paragraph]
The authors apologize for this error and state that this does not change the scientific conclusions of the article in any way.


%%% add references as needed if the correction includes referenced data.


%\bibliographystyle{frontiersinSCNS_ENG_HUMS} % for Science, Engineering and Humanities and Social Sciences articles, for Humanities and Social Sciences articles please include page numbers in the in-text citations
%\bibliographystyle{frontiersinHLTH&FPHY} % for Health and Physics articles
%\bibliography{test}

%%% Upload the *bib file along with the *tex file and PDF on submission if the bibliography is not in the main *tex file


%%% If this corrigendum relates to a figure or figures please upload it/them when submitting the article
%%% Frontiers will add the figures at the end of the provisional pdf automatically
%%% The use of LaTeX coding to draw Diagrams/Figures/Structures should be avoided. They should be external callouts including graphics.


%%%%%%%%%%%%%%%%%%%%%%%%%%%%%%%%%%%%%%%%%%%%%%%%%%%%%%%%%%%%%%%%%%%%%%%%%%%%%%%%%%%
% More examples of how to write corrigenda can be found on the following pages:
%
% http://journal.frontiersin.org/article/10.3389/fcell.2016.00074/full
% http://journal.frontiersin.org/article/10.3389/fpls.2016.01041/full
% http://journal.frontiersin.org/article/10.3389/fpls.2016.00281/full
%
%%%%%%%%%%%%%%%%%%%%%%%%%%%%%%%%%%%%%%%%%%%%%%%%%%%%%%%%%%%%%%%%%%%%%%%%%%%%%%%%%%%

\end{document}
